\documentclass{article}
% Damit die Verwendung der deutschen Sprache nicht ganz so umst\"andlich wird,
% sollte man die folgenden Pakete einbinden: 
\usepackage[utf8]{inputenc}
\usepackage{listings}
\usepackage{color}
\usepackage{blindtext}
\usepackage{graphicx}
\usepackage{subfiles}
\usepackage{amsmath}
\usepackage{tabularx}
\usepackage[thinlines]{easytable}
\usepackage{titlesec}
\newcommand{\sectionbreak}{\clearpage}

\definecolor{codegreen}{rgb}{0,0.6,0}
\definecolor{codegray}{rgb}{0.5,0.5,0.5}
\definecolor{codepurple}{rgb}{0.58,0,0.82}
\definecolor{backcolour}{rgb}{0.96,0.96,0.96}
\title{Machine Learning}
\author{Luca Schneider}

\lstdefinestyle{mystyle}{
	backgroundcolor=\color{backcolour},   
	commentstyle=\color{codegreen},
	keywordstyle=\color{magenta},
	numberstyle=\tiny\color{codegray},
	stringstyle=\color{codepurple},
	basicstyle=\footnotesize,
	breakatwhitespace=false,         
	breaklines=true,                 
	captionpos=b,                    
	keepspaces=true,                 
	numbers=left,                    
	numbersep=5pt,                  
	showspaces=false,                
	showstringspaces=false,
	showtabs=false,                  
	tabsize=2
}

\lstset{style=mystyle,language=C++}

\date{\today}


% Hinweis: \title{um was auch immer es geht}, \author{wer es auch immer 
% geschrieben hat} und  \date{wann auch immer das war} k\"onnen vor 
% oder nach dem  Kommando \begin{document} stehen 
% Aber der \maketitle Befehl mu\ss{} nach dem \begin{document} Kommando stehen! 
\begin{document}
	
	
	\maketitle
	
	\newpage
	
	\tableofcontents
	
	\newpage

	
	\section{Chapter 2 - Statistical Learning}
	
	\subfile{sections/lecture1}
	
	\section{Chapter 3 - Linear Regression}
	
	\subfile{sections/lecture2}
	
	\section{Chapter 4 - Classification}
	
	\subfile{sections/lecture3}
	
	\section{Chapter 5 - Resampling Methods}
	
	\subfile{sections/lecture4}
	
	\section{Chapter 6 - Linear Model Selection and Regularization}
	
	\subfile{sections/lecture5}
	
	\section{Chapter 7 - Moving Beyond Linearity}
	
	\subfile{sections/lecture6}
	
	\section{Chapter 8 - Tree-based Methods}
	
	\subfile{sections/lecture7}
	
	\section{Chapter 9 - Support Vector Machines}
	
	\subfile{sections/lecture8}
	
	\section{Chapter 10 - Unsupervised Learning}
	
	\subfile{sections/lecture9}
	
	\section{Important Formulas}
	
	\subfile{sections/guideline}
	
\end{document}